\documentclass{article}
\usepackage{amsmath}
\usepackage{paralist}
\title{CMPS 101, Spring 2016: HW 1}
\author{Merrick Swaffar and Siobhan O'Shea}
\date{29 April 2016}

\begin{document}
\maketitle

\begin{description}
\item[Q1:]
    \begin{itemize}
        \item a = 7, b = 3, c = 2, case 3 \\
         $T(n) \leq 7 T(n/3) + n^{2}$ \\
         $log_37 = 2$\\
         $T(n) = O(n^{log_37})$
        \vspace{5mm}    
        
        \item a = 7, b = 3, c = 1, case 1 \\
         $T(n) \leq 7 T(n/3) + n$ \\
         $log_37 > 1$\\
         $T(n) = O(n^{log_37})$
        \vspace{5mm}
        
        \item a = 7, b = 3, c = 0, case 1 \\
         $T(n) \leq 7 T(n/3) + 1$ \\
         $log_37 > 0$\\
         $T(n) = O(n^{log_37})$
        \vspace{5mm}
        
        \item Master Theorem cases: \\
        $T(n) = aT(n/b) + f(n)$\\
        
        Case 1 : \\
        $ f(n) \epsilon O(n^2), c<log_ba \Rightarrow  
        T(n) \epsilon \theta (n^{log_ba})$\\
       
        Case 2 : \\
        $ f(n) \epsilon \Theta (n^{c}log^{k}n), c = log_ba \Rightarrow 
        T(n) \epsilon \theta (n^{c}log^{k+1}n)$\\
      
        Case 3: \\
        $f(n) \epsilon \Omega (n^{c}), c > log_ba
        \Rightarrow 
        T(n) \epsilon \theta (f(n)) $      
    \end{itemize}

\vspace{5mm}
\item[Q2:] 
\begin{itemize}
$T(n) \leq 2T(n/2) + \sqrt{n}, T(1)=1$\\
Prove T(n) $\epsilon$ O(n)\\
\\
$T(n)\leq an+b\sqrt{n}\Rightarrow T(n) \epsilon O(n)$, where a and b are sufficiently large constants\\
\\
base case:\\
$T(1) = 1 \leq a(1)+b\sqrt{1}$\\
for a = 2 and b = 1\\
\\
inductive hypothesis:\\
$T(n/2) \leq a(n/2) + b \sqrt{n/2}$\\
\\
induction:\\
$T(n) \leq 2T(n/2) + \sqrt{n}$\\
$\hspace*{5mm}\leq 2(a(n/2) + b \sqrt{n/2}) + \sqrt{n}$\\
$\hspace*{5mm}\leq a(n) + \frac{2}{\sqrt{2}}b \sqrt{n} + \sqrt{n}$\\
$\hspace*{5mm}\leq a(n) + (\frac{2}{\sqrt{2}}b + 1) \sqrt{n}$\\
$\hspace*{5mm}\leq a(n) + b\sqrt{n}$\\
\end{itemize}
\vspace{5mm}

\item[Q3:] 
\begin{itemize}
\item inversions(A)\\
\hspace*{5mm}n = A.length\\
\hspace*{5mm}if n $<$ 2\\
\hspace*{10mm}return 0\\
\hspace*{5mm}L = A[1, ... , n/2]\\
\hspace*{5mm}R = A[n/2 + 1, ... , n]\\
\hspace*{5mm}Count$+$ = inversions(L)\\
\hspace*{5mm}Count$+$ = inversions(R)\\
\hspace*{5mm}i = j = k = 1\\
\hspace*{5mm}while i $<$ L.length or j $<$ R.length\\
\hspace*{10mm}if L[i] $\leq$ C[j]\\
\hspace*{10mm}i = i+1\\
\hspace*{5mm}else\\
\hspace*{10mm}j = j+1\\
\hspace*{10mm}count + = L.length - i\\
\hspace*{5mm}return count\\
\vspace{1mm}

The time complexity of this algorithm is $\Theta (nlogn)$. You are recursively splitting the array into two sub arrays, then you perform a linear time combine step. Therefore, it's time complexity is governed by the recurrence $T(n) = 2T(n/2) + cn$, just as with merge sort.
\end{itemize}
\vspace{5mm}

\item[Q4:] 
\begin{itemize}
\item[1.] 
$k^{th}$smallest(A,k)\\
\hspace*{5mm}i = partition(A)   //the partition algorithm discussed in class\\
\hspace*{5mm}if (i = k)\\
\hspace*{10mm}return A[k]\\
\hspace*{5mm}n = A.length\\
\hspace*{5mm}L = A[1, ... , i - 1]\\
\hspace*{5mm}R = A[i + 1, ... , n]\\
\hspace*{5mm}if (i $>$ k)\\
\hspace*{10mm}return $k^{th}$smallest(L, k)\\
\hspace*{5mm}return $k^{th}$smallest(R, k - i)
\vspace{2mm}
\item[2.] 
In the worst case partition separates the array into two arrays of size 0 and $n-1$. This implies that there are $\sum_{i=1}^{n} (n-1)$ calls to partition, and partition runs in linear time, so the worst case time complexity is $O(n^2)$.
\vspace{2mm}
\item[3.]
randomized-$k^{th}$smallest(A,k)\\
\hspace*{5mm}n = A.length\\
\hspace*{5mm}r = random(1 to n)\\
\hspace*{5mm}swap(A[1] , A[r])\\
\hspace*{5mm}i = partition(A)   //the partition algorithm discussed in class\\
\hspace*{5mm}if (i = k)\\
\hspace*{10mm}return A[k]\\
\hspace*{5mm}L = A[1, ... , i - 1]\\
\hspace*{5mm}R = A[i + 1, ... , n]\\
\hspace*{5mm}if (i $>$ k)\\
\hspace*{10mm}return $k^{th}$smallest(L, k)\\
\hspace*{5mm}return $k^{th}$smallest(R, k - i)\\


$T(n) \leq \frac{2}{n} [\sum_{s=1}^{n-1} T(S)] + cn$\\
$T(n) \leq anlog_2n - bn$\\
\\
base case:\\
$n=2$\\
$c = T(2) \leq a2log_22 - 2b$\\
$=2a-2b$\\
$c \leq 2(a-b)$\\
\\
induction:\\
Assume $T(s) \leq aslog_2s - bs \forall  s<n$\\
$T(n) \leq [\sum_{s=1}^{n-1} (aslog_2s - bs)]+cn$\\
$\hspace*{5mm}\leq \frac{2a}{n}[\sum_{s=1}^{n-1} (slog_2s)] - \frac{2b}{n}[\sum_{s=1}^{n-1} (s)] + cn$\\
$\hspace*{5mm}\leq anlog_2n - bn$\\
\end{itemize}
\vspace{5mm}


\end{description}
\end{document}